\documentclass[12pt]{article}
\usepackage{amsmath,amssymb}

\title{CRT and basics of modular arithmetic}

\begin{document}
	\maketitle
	\begin{enumerate}
		\item[\textbf{P1 (O1)}]
		Let \( p \) be a prime and \( a,b \in \mathbb{Z} \).
		Show that there exists
		\[
		x \in \{0,1,\dots,p-1\}
		\]
		such that
		\[
		ax + b \equiv 0 \pmod{p}.
		\]
		
		\item[\textbf{P2 (O1)}]
		Let \( p,q \) be distinct primes and \( a,b \in \mathbb{Z} \).
		Suppose there exist integers \( x_1,x_2 \) such that
		\[
		ax_1 + b \equiv 0 \pmod{p}
		\quad \text{and} \quad
		ax_2 + b \equiv 0 \pmod{q}.
		\]
		Show that there exists an integer \( x \) such that
		\[
		ax + b \equiv 0 \pmod{pq}.
		\]
		
		\item[\textbf{P3 (O3)}]
		Let \( x,y \in \mathbb{N} \) and let \( p,q \) be primes.
		Show that the equation
		\[
		(x+y)^2 = (pq+1)x + y
		\]
		has at most four solutions in natural numbers.
		
		\item[\textbf{P4 (O2)}]
		Let \( p \) be a prime and let \( a \in \mathbb{N} \) with \( p \nmid a \).
		Bob and Amy start with \( n=a \) and alternately replace \( n \) by \( nb \),
		where \( p \nmid b \), starting with Bob.
		Amy wins if she can on her turn replace current number $n$, with $m$, such that
		\[
		m \equiv 1 \pmod{p}.
		\]
		For which values of \( a \) does Amy have a winning strategy?
		
		\item[\textbf{P5 (O2)}]
		Let \( P(x) \) be a polynomial with integer coefficients and let \( q \) be a prime.
		Show that for all integers \( a \),
		\[
		P(a+q) \equiv P(a) \pmod{q}.
		\]
	\end{enumerate}
	
\end{document}
